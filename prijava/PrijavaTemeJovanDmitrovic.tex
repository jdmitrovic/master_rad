\documentclass[a4paper]{article}
\usepackage[utf8]{inputenc}
\usepackage[T2A]{fontenc}
\setlength{\textheight}{25cm}
\setlength{\textwidth}{18cm}
\setlength{\topmargin}{-25mm}
\setlength{\hoffset}{-25mm}
\def\zn{,\kern-0.09em,}

\begin{document}
\thispagestyle{empty}

\begin{flushleft}
Математички факултет\\
Универзитета у Београду
\end{flushleft}

\bigskip

\begin{center}
\textbf{МОЛБА\\
ЗА ОДОБРАВАЊЕ ТЕМЕ МАСТЕР РАДА
}\end{center}

\bigskip

\begin{flushleft}
Молим да ми се одобри израда мастер рада под насловом:
\end{flushleft}

\begin{minipage}{16.5cm}
%%%%%%%%%%%%%%%%%%%%%%%%%%%%%%%%%%%%%%%%%%%%%%%%%%%%%%%%%%%%%%%%%%%%%%%%%%%%%%%
% U donji red upisati naziv master rada umesto teksta: >>Назив мастер рада<<  %
%%%%%%%%%%%%%%%%%%%%%%%%%%%%%%%%%%%%%%%%%%%%%%%%%%%%%%%%%%%%%%%%%%%%%%%%%%%%%%%
\textbf{\textit{\zn Имплементација прилагодљивог радикс-стабла у програмском језику \emph{Rust}''}}
\end{minipage}\\
\rule[4mm]{17.5cm}{.05mm}
\begin{flushleft}
\framebox{
\begin{minipage}[10cm]{17cm}
%%%%%%%%%%%%%%%%%%%%%%%%%%%%%%%%%%%%%%%%%%%%%%%%%%%%%%%%%%%%%%%%%%%%%%%%%%%%%%%
% 	-- unutrasnjost pravougaonika --    	  								  %
%%%%%%%%%%%%%%%%%%%%%%%%%%%%%%%%%%%%%%%%%%%%%%%%%%%%%%%%%%%%%%%%%%%%%%%%%%%%%%%
\textbf{Значај теме и области:}

% 	Umesto donjeg teksta opisati značaj teme i oblasti	%

Радикс-стабла (енгл. \emph{radix trees}) су дрволика структура података која
похрањује парове кључ-вредност. Како унутрашњи чворови радикс-стабла
представљају делове кључа, а листови садрже вредности, висина стабла не
зависи од количине већ унетих података, већ искључиво од дужине
кључа. Ова особина, уз чињеницу да се у радикс-стаблима може одржавати
поредак кључева, чини радикс-стабло кандидатом за употребу у базама
података, где се може повлачити више вредности сличних кључева истовремено.
Недостатак радикс-стабла је у спрези између ефикасности и степену
искоришћења меморије. Предлог решења су прилагодљива радикс-стабла
(енгл. \emph{Adaptive Radix Tree}, скраћено ART) која
величину унутрашњих чворова могу прилагодити потреби, чиме се оптимизује
потрошња меморије.

Програмски језик \emph{Rust} је модеран и изузетно популаран језик
чија прва верзија је објављена 2015.\ године. \emph{Rust} је робустан
језик опште намене и има велики спектар могућих примена али пуна подршка
за рад са прилагодљивим радикс-стаблима и даље није доступна.\\

\textbf{Специфични циљ рада:}

% 	Umesto donjeg teksta opisati specifični cilj master rada %

Коришћењем механизама програмског језика \emph{Rust} треба имплементирати
структуру података ART тако да подржава основне операције: уметање, читање,
ажурирање и брисање. Потребно је написати тест примере који ће проверити
да ли имплементирана структура коректно обавља операције. На крају,
анализирати перформансе структуре података ART и упоредити са перформансама
других струкура података имплементираних у програмском језику \emph{Rust}. \\

\textbf{Литература:}

[1] \textbf{Viktor Leis, Alfons Kemper, Thomas Neumann}: \textit{The Adaptive Radix Tree:
  ARTful Indexing for Main-Memory Databases}, 2013

[2] \textbf{Steve Klabnik, Carol Nichols}: \textit{The Rust Programming Language}, 2019

\end{minipage}
}
\end{flushleft}
\vspace{1cm}
%%%%%%%%%%%%%%%%%%%%%%%%%%%%%%%%%%%%%%%%%%%%%%%%%%%%%%%%%%%%%%%%%%%%%%%%%%%%%%%
% u donji red uneti:       ime i prezime, broj indeksa i modul studenta       %
%%%%%%%%%%%%%%%%%%%%%%%%%%%%%%%%%%%%%%%%%%%%%%%%%%%%%%%%%%%%%%%%%%%%%%%%%%%%%%%
\makebox[8.5cm][c]{\textbf{Јован Дмитровић, 1094/2018, Информатика}}
%%%%%%%%%%%%%%%%%%%%%%%%%%%%%%%%%%%%%%%%%%%%%%%%%%%%%%%%%%%%%%%%%%%%%%%%%%%%%%%
% u donji red uneti:                   ime i prezime mentora				  %
%%%%%%%%%%%%%%%%%%%%%%%%%%%%%%%%%%%%%%%%%%%%%%%%%%%%%%%%%%%%%%%%%%%%%%%%%%%%%%%
Сагласан ментор \makebox[7.5cm][c]{\textbf{проф. др Милена Вујошевић Јаничић}} \\
\rule[4mm]{8.5cm}{.05mm} \hfill \raisebox{4mm}{\makebox[6.5cm][l]{.\dotfill.}} \\
\raisebox{1cm}%
[9mm][0mm]{\makebox[8.5cm][c]{\textit{(име и презиме студента, бр. индекса, модул)}}} \\
\makebox[10cm]{ }\\
\vspace{-1cm}\\
\rule[2cm]{6.5cm}{.05mm} \hfill \rule[2cm]{6.5cm}{.05mm}\\
\vspace{-2.4cm}\\
\raisebox{2cm}{\makebox[6.5cm][c]{\textit{(својеручни потпис студента)}}}
\hfill \raisebox{2cm}{\makebox[6.5cm][c]{\textit{(својеручни потпис ментора)}}}\\
\vspace{-2cm}\\
%%%%%%%%%%%%%%%%%%%%%%%%%%%%%%%%%%%%%%%%%%%%%%%%%%%%%%%%%%%%%%%%%%%%%%%%%%%%%%%
% u donji red uneti datum podnosenja molbe									  %
%%%%%%%%%%%%%%%%%%%%%%%%%%%%%%%%%%%%%%%%%%%%%%%%%%%%%%%%%%%%%%%%%%%%%%%%%%%%%%%
\makebox[5.5cm][c]{\textbf{25. јануар 2022.}}\makebox[5.5cm]{}  Чланови комисије\\
%%%%%%%%%%%%%%%%%%%%%%%%%%%%%%%%%%%%%%%%%%%%%%%%%%%%%%%%%%%%%%%%%%%%%%%%%%%%%%%
% POPUNJAVA MENTOR (rucno ili na sledeci nacin):							  %
% u donji red umesto .\dotfill. upisati podatke o 1. clanu komisije		      %
%%%%%%%%%%%%%%%%%%%%%%%%%%%%%%%%%%%%%%%%%%%%%%%%%%%%%%%%%%%%%%%%%%%%%%%%%%%%%%%
\rule[4mm]{5.5cm}{.05mm}\makebox[5.5cm]{ } 1. \makebox[6cm][l]{доц. др Весна Маринковић}\\
\vspace{-8mm}\\
\raisebox{4mm}%														
[7mm][0mm]{\makebox[5.5cm][c]{\textit{(датум подношења молбе)}}}\makebox[5.5cm]{ }
%%%%%%%%%%%%%%%%%%%%%%%%%%%%%%%%%%%%%%%%%%%%%%%%%%%%%%%%%%%%%%%%%%%%%%%%%%%%%%%
% POPUNJAVA MENTOR (rucno ili na sledeci nacin): 							  %
% u donji red umesto .\dotfill. upisati podatke o 2. clanu komisije           %
%%%%%%%%%%%%%%%%%%%%%%%%%%%%%%%%%%%%%%%%%%%%%%%%%%%%%%%%%%%%%%%%%%%%%%%%%%%%%%%
2. \makebox[6cm][l]{доц. др Нина Радојичић Матић}\\

\vspace{1cm}


\begin{flushleft}
%%%%%%%%%%%%%%%%%%%%%%%%%%%%%%%%%%%%%%%%%%%%%%%%%%%%%%%%%%%%%%%%%%%%%%%%%%%%%%%
% u donji red upisati              katedru									  %
%%%%%%%%%%%%%%%%%%%%%%%%%%%%%%%%%%%%%%%%%%%%%%%%%%%%%%%%%%%%%%%%%%%%%%%%%%%%%%%
Катедра \makebox[6.5cm][l]{ \textbf{за рачунарство и информатику}} је сагласна са предложеном темом.
\vspace{-3mm}
\hspace*{13mm} \rule[2.3cm]{6.5cm}{.05mm}\\
\vspace{-1cm}
%%%%%%%%%%%%%%%%%%%%%%%%%%%%%%%%%%%%%%%%%%%%%%%%%%%%%%%%%%%%%%%%%%%%%%%%%%%%%%
% POPUNJAVA SEF KATEDRE                                                      %
%%%%%%%%%%%%%%%%%%%%%%%%%%%%%%%%%%%%%%%%%%%%%%%%%%%%%%%%%%%%%%%%%%%%%%%%%%%%%%
\makebox[6.5cm][c]{} \hfill \makebox[6.5cm][c]{}\\
\rule[4mm]{6.5cm}{.05mm} \hfill \rule[4mm]{6.5cm}{.05mm}\\
\vspace{-5mm}
\makebox[6.5cm][c]{\textit{(шеф катедре)}} \hfill \makebox[6.5cm][c]{\textit{(датум одобравања молбе)}}
\end{flushleft}
\end{document}
