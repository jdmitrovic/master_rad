% Format teze zasnovan je na paketu memoir
% http://tug.ctan.org/macros/latex/contrib/memoir/memman.pdf ili
% http://texdoc.net/texmf-dist/doc/latex/memoir/memman.pdf
% 
% Prilikom zadavanja klase memoir, navedenim opcijama se podešava 
% veličina slova (12pt) i jednostrano štampanje (oneside).
% Ove parametre možete menjati samo ako pravite nezvanične verzije
% mastera za privatnu upotrebu (na primer, u b5 varijanti ima smisla 
% smanjiti 
\documentclass[12pt,oneside]{memoir} 

% Paket koji definiše sve specifičnosti master rada Matematičkog fakulteta
\usepackage[latinica]{matfmaster} 
%
% Podrazumevano pismo je ćirilica.
%   Ako koristite pdflatex, a ne xetex, sav latinički tekst na srpskom jeziku
%   treba biti okružen sa \lat{...} ili \begin{latinica}...\end{latinica}.
%
% Opicija [latinica]:
%   ako želite da pišete latiniciom, dodajte opciju "latinica" tj.
%   prethodni paket uključite pomoću: \usepackage[latinica]{matfmaster}.
%   Ako koristite pdflatex, a ne xetex, sav ćirilički tekst treba biti
%   okružen sa \cir{...} ili \begin{cirilica}...\end{cirilica}.
%
% Opcija [biblatex]:
%   ako želite da koristite reference na više jezika i umesto paketa
%   bibtex da koristite BibLaTeX/Biber, dodajte opciju "biblatex" tj.
%   prethodni paket uključite pomoću: \usepackage[biblatex]{matfmaster}
%
% Opcija [b5paper]:
%   ako želite da napravite verziju teze u manjem (b5) formatu, navedite
%   opciju "b5paper", tj. prethodni paket uključite pomoću: 
%   \usepackage[b5paper]{matfmaster}. Tada ima smisla razmisliti o promeni
%   veličine slova (izmenom opcije 12pt na 11pt u \documentclass{memoir}).
%
% Naravno, opcije je moguće kombinovati.
% Npr. \usepackage[b5paper,biblatex]{matfmaster}

\usepackage{noto}
\usepackage{listings}
\usepackage{listings-rust}
\lstdefinelanguage{TOML}{
  keywords={package, dependencies},
  keywordstyle=\color{blue}\bfseries,
  ndkeywords={class, export, boolean, throw, implements, import, this},
  ndkeywordstyle=\color{darkgray}\bfseries,
  identifierstyle=\color{black},
  sensitive=false,
  comment=[l]{//},
  morecomment=[s]{/*}{*/},
  commentstyle=\color{purple}\ttfamily,
  stringstyle=\color{red}\ttfamily,
  morestring=[b]',
  morestring=[b]"
}

\lstset{
   language=TOML,
   backgroundcolor=\color{lightgray},
   extendedchars=true,
   basicstyle=\footnotesize\ttfamily,
   showstringspaces=false,
   showspaces=false,
   numbers=left,
   numberstyle=\tiny,
   numbersep=9pt,
   tabsize=2,
   breaklines=true,
   showtabs=false,
   captionpos=b,
   xleftmargin=0.5cm,
   frame=tlbr,
   framesep=4pt,
   framerule=0pt
}

% Datoteka sa literaturom u BibTex tj. BibLaTeX/Biber formatu
\bib{matfmaster-primer}

% Ime kandidata na srpskom jeziku (u odabranom pismu)
\autor{Jovan Dmitrović}
% Naslov teze na srpskom jeziku (u odabranom pismu)
\naslov{Master iz matematike ili računarstva čiji je naslov jako dugačak}
% Godina u kojoj je teza predana komisiji
\godina{2021}
% Ime i afilijacija mentora (u odabranom pismu)
\mentor{dr Mika \textsc{Mikić}, redovan profesor\\ Univerzitet u Beogradu, Matematički fakultet}
% Ime i afilijacija prvog člana komisije (u odabranom pismu)
\komisijaA{dr Ana \textsc{Anić}, vanredni profesor\\ University of Disneyland, Nedođija}
% Ime i afilijacija drugog člana komisije (u odabranom pismu)
\komisijaB{dr Laza \textsc{Lazić}, docent\\ Univerzitet u Beogradu, Matematički fakultet}
% Ime i afilijacija trećeg člana komisije (opciono)
% \komisijaC{}
% Ime i afilijacija četvrtog člana komisije (opciono)
% \komisijaD{}
% Datum odbrane (odkomentarisati narednu liniju i upisati datum odbrane ako je poznat)
% \datumodbrane{}

% Apstrakt na srpskom jeziku (u odabranom pismu)
\apstr{Apstrakt ide ovde}

% Ključne reči na srpskom jeziku (u odabranom pismu)
\kljucnereci{programiranje, programski jezici}

\begin{document}
% ==============================================================================
% Uvodni deo teze
\frontmatter
% ==============================================================================
% Naslovna strana
\naslovna
% Strana sa podacima o mentoru i članovima komisije
\komisija
% Strana sa posvetom (u odabranom pismu)
\posveta{Mami, tati i dedi}
% Strana sa podacima o disertaciji na srpskom jeziku
\apstrakt
% Sadržaj teze
\tableofcontents*

% ==============================================================================
% Glavni deo teze
\mainmatter
% ==============================================================================

% ------------------------------------------------------------------------------
\chapter{Uvod}
% ------------------------------------------------------------------------------
\chapter{Programski jezik \emph{Rust}}
\emph{Rust} je statički tipiziran jezik koji ima podršku za više programskih 
paradigmi, fokusiran na bezbednost i performanse. Od svog nastanka, ovaj jezik
je dobio veliku pažnju u svetu programiranja, čemu svedoči i činjenica da je 
\emph{Rust} bio proglašen za "omiljeni programski jezik" već petu godinu za redom 
u anketi koju je sproveo popularni veb-sajt \emph{Stack Overflow}.

Danas se \emph{Rust} koristi na sve većem broju ozbiljnih projekata, na primer:

\begin{itemize}
    \item AWS servisima firme Amazon, poput \emph{Lambda}, \emph{EC2}
        i \emph{Cloudfront},
    \item U okviru operativnog sistema kompanije Gugl (engl. \emph{Google}) 
        \emph{ChromeOS},
    \item Određenim komponentama Majkrosoftove platforme \emph{Azure}, uključujući i 
        IoT sigurosni servis \emph{edgelet},
    \item Registru \emph{JavaScript} paketa \emph{npm}, 
        kod procedura koje prouzrokuju veliko CPU opterećenje,
    \item Mozilinom veb-brauzeru Fajerfoks (engl. \emph{Firefox}).
\end{itemize}

\section{Istorijat}
Programski jezik \emph{Rust} je dizajnirao Grejdon Hor 
(engl. \emph{Graydon Hoare}) koji je, u to vreme, bio zaposlen u kompaniji 
Mozila (engl. \emph{Mozilla}).
Hor je rad na ovom jeziku započeo 2006. godine kao svoj lični projekat, 
na kojem je samostalno radio naredne tri godine.
Kada je \emph{Rust} počeo da zreli, tada se u projekat 
uključila i sama Mozila, koja i dan-danas sponzoriše njegov razvoj.
Pored zaposlenih Mozile, pošto je u pitanju programski jezik otvorenog koda,
svoj doprinos je dalo i preko 5000 dobrovoljaca.

Pre nego što se Mozila priključila projektu, \emph{Rust} je izgledao dosta 
drugačije nego danas. U ovoj, početnoj, fazi, \emph{Rust} je bio čist jezik, 
tj. nije imao bočne efekte; takođe, postojala je i analiza stanja tipa 
(engl. \emph{typestate analysis}), koja je omogućavala proveru operacija 
koje se mogu izvoditi nad specifičnim tipom podataka pri kompajliranju. Za 
razliku od ove dve osobine, neka dizajnerska rešenja su ostala do danas, kao 
što je imutabilnost i kontrola pristupa memoriji.

Nedugo nakon priključivanja Mozile 2010. godine, Grejdon Hor napušta projekat 
2012. godine, što dovodi do određenih novina; \emph{Rust} dobija svoj menadžer 
paketa \emph{Cargo}, kao i sakupljač otpadaka. Proces \emph{RFD}, inspirisan 
procesom \emph{PEP} programskog jezika \emph{Python}, se osniva 2014. 
godine u svrhe strogog kontrolisanja novina u samom jeziku.

\emph{Rust} 1.0, prva "stabilna" verzija \emph{Rust}-a je distribuirana 2015. 
godine. Od tad, \emph{Rust} ima politiku izbacivanja novih 
verzija gde se one distribuiraju svakih 6 nedelja, što je "agresivniji" pristup 
u odnosu na većinu programskih jezika gde je taj period minimalno godinu dana. 
Ovom odlukom se stavlja akcenat na stabilnost jezika time što će svaka nova 
verzija biti slična svom prethodniku, dok se kod jezika sa dugim periodom 
između verzija očekuju velike promene, što može da šteti kompatibilnosti. 

\section{Instalacija}
Ukoliko se zvaničan veb-sajt programskog jezika \emph{Rust} poseti na 
klijentu koji ima \emph{Windows} operativni sistem, biće ponuđeni 
instalacioni fajlovi za 32-bitne i 64-bitne sisteme.

Na \emph{GNU/Linux} operativnim sistemima potrebno je uneti sledeću 
komandu u terminal:

\begin{verbatim}
curl --proto '=https' --tlsv1.2 https://sh.rustup.rs 
 -sSf | sh
\end{verbatim}

Potvrdu da li se instalacija izvršila uspešno može se dobiti 
komandom:

\begin{verbatim}
rustc --version
\end{verbatim}

Pored samog \emph{Rust} kompajlera, u instalaciju su uključeni i 
\emph{Cargo} i \emph{rustup}; \emph{rustup} daje mogućnost 
dobavljanja nove verzije \emph{Rust}-a sa veba, kao i 
mogućnost deinstalacije komandama:

\begin{verbatim}
rustup update
rustup self uninstall
\end{verbatim}

\section{Korišćenje sistema \emph{Cargo}}
Pored toga što je menadžer paketa, \emph{Cargo} vrši i automatizaciju 
kompajliranja. Kreiranje novog projekta uz pomoć ovog sistema izvršava 
se komandom:

\begin{verbatim}
cargo new novi_projekat
\end{verbatim}

Komandom iznad se pravi novi direktorijum \texttt{novi\_projekat} koji 
sadrži fajl \texttt{Cargo.toml} i direktorijum \texttt{src} u kome 
postoji fajl \texttt{main.rs}. Takođe, sa novim direktorijumom se 
inicijalizuje i novi \emph{Git} repozitorijum.

Generisani \texttt{Cargo.toml} fajl izgleda ovako:

\lstinputlisting[language=TOML]{Cargo.toml}

U prvoj liniji koda, \texttt{[package]} označava sekciju koja opisuje 
paket koji je napravljen; informacije koje se ovde nalaze su dobijene 
iz varijabli okruženja. Posle oznake \texttt{[dependencies]} se 
popisuju svi paketi koji su neophodni za rad sa novim paketom, 
tako da ih \emph{Cargo} može dopremiti.

Po osnovnim podešavanjima, u \texttt{main.rs} fajlu se nalazi 
\emph{Hello World} program.

\lstinputlisting[language=Rust]{main.rs}

\emph{Cargo} takođe može kompajlirati projekat komandom 
\texttt{cargo build} ili ga kompajlirati i pokrenuti sa 
\texttt{cargo run}. Korišćenjem komande \texttt{cargo check} 
može se proveriti da li se kod kompajlira, bez generisanja 
izvršnog fajla, što je korisno jer je ova opcija efikasnija 
od korišćenja pomenute \texttt{build} komande. Kompajliranjem 
projekta se pravi nova putanja \texttt{target/debug}, gde će se 
generisati izvršni fajlovi.

\section{Osnovne karakteristike}
U programskom jeziku \emph{Rust}, kada se nova promenljiva definiše 
ključnom rečju \texttt{let}, podrazumevano ponašanje je da je ta 
promenljiva imutabilna, tj. ona se ne može menjati. Razlog ovakvog 
ponašanja leži u tome što \emph{Rust} teži tome da kompajler može 
prepoznati eventualne greške u kodu, koje bi se teško mogle uočiti 
ukoliko bi se one dešavale tokom izvršavanja programa. I pored 
eventualnih problema, mutabilne promenljive se mogu definisati 
sa \texttt{let mut}.

Moguće je definisati i konstante ključnom rečju \texttt{const}. 
Konstante se razlikuju od imutabilnih promeljivih po tome što se 
mogu definisati u bilo kom opsegu, uključujući i globalni, i po 
tome što konstante samo mogu imati vrednost konstantnog izraza, 
ali ne i vrednost izvršavanja funkcije.

Još jedna od opcija je i tzv. sakrivanje (engl. \emph{shadowing}). 
Sakrivanje je ponovno definisanje promenljivih.

% \section{Primeri korišćenja klasičnih \LaTeX{} elemenata}
% % Primeri citiranja
% Ovo je rečenica u kojoj se javlja citat \cite{PetrovicMikic2015}.
% Još jedan citat \cite{GuSh:243}.
% % Primeri navodnika
% Isprobavamo navodnike: "Rekao je da mu se javimo sutra".
% % Primer referisanja na tabelu (koja se javlja kasnije)
% U tabeli \ref{tbl:rezultati} koja sledi prikazani su rezultati eksperimenta.
% % Primer kraćeg ćiriličkog teksta
% {\cir Ово је пример ћириличког текста који се јавља у латиничком документу.}
% U ovoj rečenici se javlja jedna reč na {\cir ћирилици}.
% % Primer korišćenja fusnota
% Iza ove rečenice sledi fusnota.\footnote{Ovo je fusnota.}

% % Primer dužeg ćirličkog teksta
% \begin{cirilica}
%   Ово је мало дужи блок текста исписан ћириличким писмом у оквиру
%   латиничког документа. Фијуче ветар у шибљу, леди пасаже и куће иза
%   њих и гунђа у оџацима.
% \end{cirilica}

% % Primer korišćenja tabele
% \begin{table}
% \centering
% \caption{Rezultati}
% \label{tbl:rezultati}
% \begin{tabular}{c>{\centering}p{2cm}c}
% \toprule
% 1 & 2 & 3\\\midrule
% 4 & 5 & 6\\\cmidrule(rl){1-2}
% 7 & 8 & 8\\
% \bottomrule
% \end{tabular}
% \end{table}

% % Primer korišćenja slike
% \begin{figure}[!ht]
%   \centering
%   \label{fig:grafikon}
%   \includegraphics[width=0.5\textwidth]{graph.png}
%   \caption{Grafikon}
% \end{figure}


% % Primer jednostavnije matematičke formule
% Evo i jedan primer matematičke formule: $e^{i\pi} + 1 = 0$. 
% % Primer referisanja na sliku
% Na slici \ref{fig:grafikon} prikazan je jedan grafikon.

% % primer kompleksnije matematičke formule
% $$
% \int_a^b f(x)\ \mathrm{d}x \ =_{def}\ \lim_{\max{\Delta x_k \rightarrow 0}} \sum_{k=1}^n f(x_k^*)\Delta x_k
% $$

% % primer referisanja na poglavlja i strane poglavlja
% Više detalja biće dato u glavi \ref{chp:razrada} na strani \pageref{chp:razrada}.

% % primer liste
% Možemo praviti i nabrajanja:
% \begin{enumerate}
% \item Analiza 1
% \item Linearna algebra
% \item Analitička geometrija
% \item Osnovi programiranja
% \end{enumerate}

% ------------------------------------------------------------------------------
\chapter{Razrada}
\label{chp:razrada}

% ------------------------------------------------------------------------------
\chapter{Zaključak}

% ------------------------------------------------------------------------------
% Literatura
% ------------------------------------------------------------------------------
\literatura

% ==============================================================================
% Završni deo teze i prilozi
\backmatter
% ==============================================================================

% ------------------------------------------------------------------------------
% Biografija kandidata
\begin{biografija}
  \textbf{Jovan Dmitrović} (\emph{Gornji Milanovac, 17.11.1995.}) 
\end{biografija}
% ------------------------------------------------------------------------------

\end{document}
